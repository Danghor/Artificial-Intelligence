\chapter{Introduction}
Artificial Intelligence has evolved through two primary approaches. The first, known as \blue{symbolic AI}, focuses on \blue{symbolic logic}. \index{symbolic AI} This approach led to the creation of automatic theorem provers, \blue{symbolic integration} systems, and chess-playing programs like \href{https://en.wikipedia.org/wiki/Deep_Blue_(chess_computer)}{Deep Blue}. Initially, symbolic AI was the predominant paradigm in the field.

The second approach, \blue{machine learning}, was defined by Arthur Samuel as ``the field of study that enables
computers to learn without explicit programming'' \cite{samuel:1959}. \index{machine learning, definition} This
approach has primarily fueled the recent hype in AI. 

\section{Overview}
This lecture discusses only symbolic AI, as machine learning is part of the module \blue{data mining}.  
It emphasizes \textcolor{blue}{declarative programming}. The core principle of declarative programming
involves starting with a \textcolor{blue}{formal problem 
specification}, a succinct description of the issue at hand. This specification is then processed by a
\textcolor{blue}{problem solver} to produce a solution. Originally,
\href{https://en.wikipedia.org/wiki/Declarative_programming}{declarative programming} adopted a broad
approach to problem-solving, where problems were framed as logical formulas and tackled using
\href{https://en.wikipedia.org/wiki/Automated_theorem_proving}{automated theorem provers}. The
programming language \href{https://en.wikipedia.org/wiki/Prolog}{Prolog} is based on this
paradigm. However, this approach has proven to be less effective as a universal problem-solving framework
for two reasons: 
\begin{enumerate}
\item It is often challenging to fully articulate practical problems within a logical framework.
\item In cases where it is possible to completely define a problem using logical formulas, automatic
       theorem proving generally lacks the capability to autonomously find solutions. 
\end{enumerate}
Despite these limitations, declarative programming has proven valuable in several domains, which we will
explore, demonstrating its application in solving various types of problems: 
\begin{enumerate}
\item \textcolor{blue}{Search problems}, where the objective is to find a path within a graph. A classic
      instance is the \href{https://en.wikipedia.org/wiki/15_puzzle}{fifteen puzzle}. We will examine several
      advanced algorithms designed to resolve such search problems. 
\item \href{https://en.wikipedia.org/wiki/Constraint_satisfaction_problem}{Constraint satisfaction
      problems} hold significant practical relevance. Currently, highly efficient constraint solvers exist,
      capable of addressing various practical constraint satisfaction problems. We will delve into different
      strategies for solving these problems and discuss \href{https://github.com/Z3Prover/z3}{Z3}, a
      leading-edge automatic theorem prover and constraint solver developed by
      \href{https://www.microsoft.com/en-us/research/project/z3-3/}{Microsoft}. 
\item \textcolor{blue}{Games}, such as \href{https://en.wikipedia.org/wiki/Chess}{chess} or
      \href{https://en.wikipedia.org/wiki/checkers}{checkers}, can be defined using a declarative
      approach. We will cover several techniques enabling computers to devise optimal strategies for these
      adversarial games. 
\item Finally, we discuss \textcolor{blue}{automatic theorem proving}. Having previously covered
      \textcolor{blue}{resolution theorem proving} in our lecture on
      \href{https://github.com/karlstroetmann/Logic}{logic}, we will now turn our attention to
      \textcolor{blue}{equational theorem proving} in the final chapter of this first part. 
\end{enumerate}

\section{Literature}
My main sources for these lecture notes were the following:
\begin{enumerate}
    \item A specialized course on artificial intelligence available through the \textsc{edX} platform. All
          relevant course materials can be accessed at
          \href{http://ai.berkeley.edu/home.html}{http://ai.berkeley.edu/home.html}. 
    \item The book titled
          \href{https://www.amazon.de/Artificial-Intelligence-Modern-Approach-Global/dp/1292401133/}{\textit{Introduction to Artificial Intelligence}},
          authored by Stuart Russell and Peter Norvig \cite{russell:2020}. 
    \item The \href{https://www.udacity.com/course/intro-to-artificial-intelligence--cs271}{\textit{Intro to
          Artificial Intelligence}} course provided by the \href{https://www.udacity.com}{Udacity} platform. 
\end{enumerate}
For exam preparation, a thorough understanding of the material covered in these lecture notes should
suffice. Therefore, purchasing additional books or enrolling in other courses is certainly not necessary.

\remark
The programs presented in these lecture notes are expected to run with the \textsl{Python} version 3.12.
I have created the Python environment that I am using for these lecture notes via the shell commands shown in
Figure \ref{fig:ai.sh} on page \pageref{fig:ai.sh}.

\begin{figure}[!ht]
\centering
\begin{minted}[ frame         = lines, 
                 framesep      = 0.3cm, 
                 firstnumber   = 1,
                 bgcolor       = bg,
                 numbers       = left,
                 numbersep     = -0.2cm,
                 xleftmargin   = 0.3cm,
                 xrightmargin  = 0.3cm,
                 ]{bash}
    conda create -y -n ai python=3.12
    conda activate ai
    pip install --upgrade pip
    pip install notebook jupyterlab
    pip install nbclassic
    conda install -c anaconda -y graphviz ply seaborn scikit-learn 
    conda install -c conda-forge -y python-graphviz matplotlib ipycanvas 
    pip install nb_mypy
    pip install z3-solver
    pip install git+https://github.com/reclinarka/problem_visuals
    pip install git+https://github.com/reclinarka/chess-problem-visuals
\end{minted}
\vspace*{-0.3cm}
\caption{Bash commands to set up an Anaconda environment for Python.}
\label{fig:ai.sh}
\end{figure}

When starting \textsl{Jupyter notebooks} you should take care to use the following command:
\\[0.2cm]
\hspace*{1.3cm}
\texttt{jupyter nbclassic}
\\[0.2cm]
This command uses the classic version of \textsl{Jupyter notebooks}.  There are lots of incompatibilites with
tne new \textsl{Jupyter notebooks} of version $7.x$ and I have found that new version $7.x$ do not work for me.


%%% Local Variables:
%%% mode: latex
%%% TeX-master: "artificial-intelligence"
%%% End:
