\chapter{Prompt Engineering}
One of the properies that clearly distinguishes humans from aninmals is the fact that humans have developed
highly sophisticated languages.  Therefore the understanding of human language is clearly one of the most
important goals of artificial intelligence.  It is obvious now that LLMs (large language models) like
\href{https://gemini.google.com/}{Gemini} or \href{https://chat.openai.com/}{ChatGPT} have achieved a degree of
human language understanding that makes them useful tools in practice.  Therefore this last chapter, which
is work in progress, discusses the question how these tools can be used most efficiently.

\section{A Motivating Example}
In my observation, the quality of answers computed by LLMs varies greatly with the quality of the input that is
given to these LLMs.  This input is commonly known as a \blue{prompt}.  To stress this point I want to give
one introductory example.  Let us assume that we want to solve the following puzzle with the help of a large language model:

\begin{center}
\colorbox{green}{\framebox{\colorbox{orange}{{\begin{minipage}{0.9\linewidth}
  I have as many brothers as I have sisters, but my sister has twice as many brothers as she has sisters.  How
  many children does my father have?  
\end{minipage}}}}}
\end{center}


If we blindly use this text as a prompt for  \textsl{Gemini Advanced} we will get the following reponse:

\begin{center}
\colorbox{red}{\framebox{\colorbox{sepia}{{\begin{minipage}{0.9\linewidth}
\begin{itemize}
\item Here's how to solve this puzzle. Let's break it down:
\item \textbf{Statement 1}: "I have as many brothers as I have sisters"
  
      This means you are one of several siblings with an equal number of boys and girls.
\item \textbf{Statement 2}: "My sister has twice as many brothers as she has sisters"

      This implies that your sister is one of the girls from the first statement. To have twice as many
      brothers as sisters, she m ust have one sister (herself).

\item \textbf{Calculation}: To figure out the total number of children, consider all the siblings:
  \begin{enumerate}
  \item You and your sister (from statement 1) = 2 children
  \item Your sister's brothers (from statement 2): She has twice as many brothers as sisters (1), so she has 2 brothers.
  \end{enumerate}
  Therefore, your father has a total of 4 children (you, your sister, and her 2 brothers).
\end{itemize}
\end{minipage}}}}}
\end{center}

Obviously, this answer is mostly garbage.  How can we improve this response?  First, we should give all the
information that there is to the LLM.  When I tell you that I have as many brothers as sisters, then you know that I am
male.  Gemini has no such knowledge.  Although the fact that I am male can be concluded from the puzzle, this
complicates the puzzle unnecessarily.  Furthermore, we can use several \blue{prompt patterns} to improve
the answer.
\begin{enumerate}
\item We can ask Gemini to, first, provide two different formalisations of the problem and then, secondly, to
      prompt us and let us choose with choose the most appropriate formalisation.
\item We can also ask Gemini to verify its solution and to explain the solution step by step.
\end{enumerate}

If we proceed along these lines, the prompt takes the following form:
\begin{center}
  \colorbox{green}{\framebox{\colorbox{orange}{{\begin{minipage}{0.9\linewidth}
            In the following, I will ask you to solve text problems.  Before proceeding to solve the problem,
            you should provide two different formalisations and should then ask me to choose from them. You
            should then solve the problem with the chosen approach and verify your solution.  If the
            verification fails, you should try the alternative approach. 
            
            I am male and I have as many brothers as I have sisters, but my sister has twice as many brothers as she has
            sisters.  How  many children does my father have?  
\end{minipage}}}}}
\end{center}

If I use this prompt, Gemini's first response provides two formalisations:

\begin{center}
  \colorbox{red}{\framebox{\colorbox{sepia}{{\begin{minipage}{0.9\linewidth}
            Here are two ways we can set up equations to represent the information in this word problem:

            \begin{description}
            \item[Formalisation 1:] \hspace*{\fill} \\
                \vspace*{-0.8cm}
              
                \begin{itemize}
                \item Let $b$ represent the number of brothers in the family (including the speaker).
                \item Let $s$ represent the number of sisters in the family.
              \end{itemize}
            \item[Formalisation 2:] \hspace*{\fill} \\
              \vspace*{-0.8cm}
              
              \begin{itemize}
              \item Let $x$ represent the total number of boys in the family.
              \item Let $y$ represent the total number of girls in the family.
              \end{itemize}

            \item[Which formalisation would you like to use to solve the problem?]
\end{description}
\end{minipage}}}}}
\end{center}

I have chosen the second formalisation, since the statement 
\\[0.2cm]
\hspace*{1.3cm}
``Let $b$ represent the number of brothers in the family (including the speaker).''
\\[0.2cm]
from the first formalisation is clearly problematic since I am not my own
brother.  After choosing the second formalisation I got the following response:

\begin{center}
  \colorbox{red}{\framebox{\colorbox{sepia}{{\begin{minipage}{0.9\linewidth}
            \begin{itemize}
            \item[] Excellent! Here's how we can solve the problem using the second formalisation:
            \item \textbf{Setting up the equations:}
            \item \textbf{Speaker's perspective:}
                   Since the speaker is male, he has $(x - 1)$ brothers. He has
                   $y$ sisters. We're told these quantities are equal: 
                   \\[0.2cm]
                   \hspace*{1.3cm}
                   $x - 1 = y$
            \item \textbf{Sister's perspective:}
                   She has $x$ brothers and $(y - 1)$ sisters. She has twice as many brothers as sisters:
                   \\[0.2cm]
                   \hspace*{1.3cm}
                   $x = 2(y - 1)$
            \item \textbf{Solving the equations}:
              \begin{enumerate}
              \item \textbf{Substitute}: We can substitute the value of $x$ from the second equation into the first:
              \item $2(y - 1) - 1 = y$
              \item $2y - 2 - 1 = y$
              \item $y = 3$
            \item \textbf{Find} $x$:  Now, substitute the value of $y$ back into either of the original
                   equations. Let's use the first one: 
            \item $x - 1 = 3$
            \item $x = 4$
          \end{enumerate}
          \item \textbf{Solution}:

            The family has 4 sons $(x)$ and 3 daughters $(y)$.

            
          \item \textbf{Verification}: 
            Let's check if this solution fits the original statements:
            \begin{enumerate}
            \item \textbf{Speaker's perspective}: The speaker has 3 brothers and 3 sisters, satisfying the condition.
            \item \textbf{Sister's Perspective}: The sister has 4 brothers and 2 sisters, satisfying the
                  condition of having twice as many brothers as sisters. 
            \item \textbf{Since the solution fits both perspectives, it's validated}.

           \end{enumerate}
         \item \textbf{Therefore, the father has 7 children.}
         \end{itemize}
\end{minipage}}}}}
\end{center}

This time Gemini has solved the problem correctly and, furthermore, it has given a clear explanation of the solution.
Therefore, the second answer is much better that the first one.  This example shows the necessity for us to craft high
quality prompts.  In this respect, LLMs are very similar to the rest of computer programs.  The follow the same
principle:
\begin{center}
  \colorbox{red}{\framebox{\colorbox{bg}{{
  Garbage in, garbage out!}}}}
\end{center}

My last prompt did use several \blue{prompt design patterns}.  We will discuss some of these in the next section.

\section{Prompt Design Patterns}
As there are \blue{Software Design Patterns} in software engineering, there are \blue{prompt design patterns}
in prompt engineering.  I have taken these patterns from the text book
\textsl{Prompt Design Patterns: Mastering the Art and Science of Prompt Engineering} by Yi Zhou \cite{zhou:2023}
and the article by Jules White et.~al.~\cite{white:2023}, which introduces a catalogue of prompt design patterns.


%%% Local Variables:
%%% mode: latex
%%% TeX-master: "artificial-intelligence"
%%% End:
